\documentclass[a4paper,9pt]{article}
\usepackage{conference}
\usepackage{latexsym}
%\usepackage[utf8]{inputenc}
\usepackage[utf8x]{inputenc}
\usepackage{textcomp}
\usepackage[english]{babel}
\usepackage{amssymb,amsfonts,amsmath}
\usepackage{graphicx}
\usepackage{cite}
% \usepackage{hyperref}
\usepackage[varg]{txfonts}
\usepackage{booktabs}       % professional-quality tables
%\usepackage[letterpaper, landscape, lmargin=0.25in, rmargin=1.25in]{geometry}
\usepackage{tabularx}
\usepackage{enumerate}
%\usepackage{enumitem}
\usepackage{multicol}
% \usepackage{ifpdf}
% \usepackage{tikz}
% \usetikzlibrary{shapes}
% \usepackage[left=0.1cm, right=0.1cm, top=0.1cm, bottom=0.1cm]{geometry}
% \usepackage[ansinew]{inputenc}            % Input
\usepackage[T1]{fontenc}                  % Font encoding
% \usepackage{cmbright}                     % Font style
%\usepackage[nohead,margin=0mm]{geometry}  % Page & margins
\usepackage{pstricks,pst-node}            % Pstricks


\usepackage{setspace}
\onehalfspacing

\usepackage{url}
\usepackage[hidelinks]{hyperref}

\hypersetup{breaklinks=true}



\pagenumbering{gobble}
%%\title{Hubble Table}
%\author{author}
\date{}
%\title{Hubble Table \emph{conference} preprint}
\title{Holographic Principle and Science Unification - holoexpo 2020}
\author{
  F.M. Sanchez
  Department of Physics\\
  Paris 11 University\\
  Paris, FRANCE \\
  \texttt{hol137@yahoo.fr} \\
  %% examples of more authors
   \And
 M.H. Grosmann \\
  Department of Photonics\\
  University of Strasbourg\\
  Strasbourg, FRANCE \\
  \texttt{michel.grosmann@me.com} \\
   \And
 L. Gueroult\\
  Department of Physique A2 \\ 
  ENS \\
  Cachan, FRANCE \\
  \texttt{l.gueroult@hotmail.com} \\ 
  %% Coauthor \\
  %% Affiliation \\
  %% Address \\
  %% \texttt{email} \\
}



\begin{document}
\maketitle

\begin{abstract}

There are generally two methods in Physics, differential equations and holistic ones.  Belonging to this second type, an Holographic Principle was introduced in Theoretical Physics, establishing bridges between theories implying different dimensions. But this principle cannot apply directly in standard cosmology, since the Hubble radius is considered as variable.  By adopting the opposite hypothesis, an invariant Hubble radius, this allows to explain the Large Number correlations, and leads to a Topological Axis which rehabilitates the bosonic tachyonic string theory. The relations are so constrived that this permits to show up 223 formulas for the Hubble radius, with 18 in the ppb range, using an optimal value for G deduced from the non-Doppler quasar oscillation period. The Planck wall is reduced by a factor $10^{61}$, and a Cosmos exceed the Hubble radius by the same factor, but the mean length remains connected with biological parameters.  
 

\end{abstract}


% keywords can be removed
\keywords{Quantum Physics \and Number theory \and Cosmology \and Holography \and Crystallography}

\clearpage

\label{sec:headings}

\section{Introduction}
. 
       

\section{The scope and method}
.
   


\section{The cosmic liaison between $a$ and the weak mixing angle}

.
  
\section{The Rehabilitation of Wyler 's theory}
. 

\section{The Central Role of the Modular Number $j_0$ = 744}

.

\section{The Decisive Role of the term $a^a$ }

.


\section {The Koide-Wyler ppb relation}

.

\section{The rehabilitation of 137 from the electroweak coupling constant}

.

\section{The Planck law connection with the Bernouilli function}
.


\section{The Holographic Fine-Tuning with the Universal Critical Radius}
.
    
\section{The connection with Diophantine and Eddington physics}


\section{The specified connection with the topological}


\section{The connection with the Periodic Table of Elements}

.
     
\section{Connections with the high-dimensional crystallography}
 


\section{Conclusion : towards an unified Science}

.

\section {Aknowledgements}

.

\bibliographystyle{unsrt}  
%\bibliography{references}  %%% Remove comment to use the external .bib file (using bibtex).
%%% and comment out the ``thebibliography'' section.
%%% Comment out this section when you \bibliography{references} is enabled.
\begin{thebibliography}{99}
\bibitem{Hirzebruch} Hirzebruch F. Topological methods in algebraic geometry. Springer 1966.\\
\bibitem{Bott} M. Atiyah, R. Bott, V. Patodi, "On the heat equation and the index theorem" Invent. Math. , 19 (1973) pp. 279--330.\\
\bibitem{Singer} M. Atiyah, I. Singer, "The index of elliptic operators IV" Ann. of Math. , 93 (1971) pp. 119--138. \\
\bibitem{Alvarez} L. Alvarez-Gaume, "Supersymmetry and the Atiyah Singer index theorem" Comm. Math. Phys. , 90 (1983) pp. 161--170.\\
\bibitem{Atiyah} Atiyah M. https://hitsmediaweb.h-its.org/Mediasite/Play/35600dda1dec419cb4e99f706197a3951d. \\ 
\bibitem{Eddington} Eddington A, Fundamental Theory, Cambridge.\\
\bibitem{Sanchez} F.M. Sanchez, V. Kotov, M. Grosmann, D. Weigel, R. Veysseyre, C. Bizouard, N. Flawisky, D. Gayral, L. Gueroult, Back to Cosmos.\\
\bibitem{Bastin} Bastin T. and Kilmister C.W., Combinatorial Physics (World Scientific, 1995).\\
\bibitem{Tanabashi} Tanabashi M. et al. (Particle Data Group), Phys. Rev. D98, 030001 (2018), and 2019 update.\\
\bibitem{Atiyah1} Atiyah M. Private Communication (december 2018).\\
\bibitem{Wyler} Wyler A., "L'espace symetrique du groupe des equations de Maxwell" C. R. Acad. Sc. Paris, t. 269, 743-745 (1969). Wyler A., C.R. Acad. Sci, Paris "Les groupes des potentiels de Coulomb et de Yukawa". C. R. Acad. Sc. Paris, t. 272, 186-188 (1971).\\
\bibitem{Conway} Conway, John Horton; Norton, Simon P. (1979). "Monstrous Moonshine". Bull. London Math. Soc. 11 (3): 308--339.\\
\bibitem{Borcherds} Borcherds, Richard (1992), "Monstrous Moonshine and Monstrous Lie Superalgebras", Invent. Math., 109: 405--444.\\
\bibitem{Sanchez1}  Sanchez F.M., Holic Principle, Entelechies, ANPA 16, Sept. 1995. Bowden K.G., 324--343.\\
\bibitem{Shannon} Shannon C.E. « A Mathematical Theory of Communication » Reprinted with corrections from The Bell System Technical Journal, Vol. 27, p. 379–423, 623–656, July, October, 1948.)\\
\bibitem{Stark} Stark H.M. A complete determination of the complex quadratic fields of class-number one, Michigan Math. J., vol. 14,‎ 1967, p. 1-27  \\
\bibitem{Lovelace} Lovelace C. (1971) Pomeron form factors and dual Regee cuts, Physics Letters B34 (6) 500-506.\\
\bibitem{Apostol} Apostol T. Modular functions and Dirichlet Series in Number Theory. Springler-Verlag. New-York (1990).\\
\bibitem{Green} Green, M. Schwarz J. (1984)  Anomaly cancellations in supersymmetric D = 10 gauge theory and superstring theory". Physics Letters B. 149: 117.\\
\bibitem{Shlay} Shray J. (1994) Octonions and Supersymmetry, PhD thesis.  http://ir.library.oregonstate.edu/xmlui/handle/1957/35649. \\
\bibitem{Koide} Koide Y., Fermion-Boson Two-Body Model of Quarks and Leptons and Cabibbo Mixing  Lett. Nuovo Cimento 34, 201 (1982). 
\bibitem{Hooft} Hooft 't Th Holographic Principle. ArXiv: hep-th/003004 (2000). \\
\bibitem{Bousso} Bousso R., The Holographic Principle, Review of Modern Physics, vol 74, p.834 (2002).\\
\bibitem{Friedman} Friedman W. et al, The Carnegie-Chicago Hubble Program. VIII. An Independent Determination of the Hubble Constant Based on the Tip of the Red Giant Branch, arxiv : 1907.05922.\\ 
\bibitem{Durham} Durham I.T. 2006, Sir Arthur Eddington and the Foundations of Modern Physics arXiv:quant-ph/0603146v1  p.111.
\bibitem{Sanchez2} Sanchez F.M., Kotov V. and Bizouard C., 'Towards a synthesis of two cosmologies: the steady- state flickering Universe'. Journal of Cosmology, vol 17. (2011).\\
\bibitem{Quinn} Quinn T, Speake C, Parks H, Davis R. 2014 The BIPM measurements of the Newtonian constant of gravitation, G. Phil.Trans. R. Soc. A372: 20140032. http://dx.doi.org/10.1098/rsta.2014.0032. \\
\bibitem{Sternheimer} Sternheimer J., Musique des particules elementaires, CRAS, 297, II, 829--834 (1983).\\
\bibitem{Weigel} Veysseyre R., Veysseyre H., and Weigel D. Counting, and Symbols of Cristallographic Point Symmetry Operations of Space En. AAECC 5, 53--70 (1994).\\
\bibitem{Carr} Carr B.J. and Rees M. J. , “The anthropic principle and the structure of the physical world”, Nature 278, 605-612 (1979).\\
\bibitem{Sanchez3} F.M. Sanchez. Coherent Cosmology Vixra.org,1601.0011. Springer International Publishing AG 2017. A. Tadjer et al. (eds.), Quantum Systems in Physics, Chemistry, and Biology, Progress in Theoretical Chemistry and Physics 30, pp. 375--407. \\ 
\end{thebibliography}






% \section{Periodic Table}
%^^^^^^^^^^^^^^^
% \input{periodic-table}

%\listoffigures{Figures}   % table of figures

%\subsection{Figure}

\begin{figure}
\label{fig:7:fig1}
% \label{fig:figure_label}
\centering
\includegraphics[width=\textwidth,height=14cm]{./figure/figure}
\caption{\textit{The Topological Axis} (data in Table 1). The double natural logarithms y = ln(ln(Y)) of the main dimensionless physical quantities (Y) corresponds to the special string dimension series d = 4k + 2, from k = 0 to k = 7, characteristics of the Bott sequence . This is the reunion of height 2D-1D holographic relations, hence the name `Topological Axis`. Two relations comes from the double large number correlation \cite{Eddington}, one comes from the Carr and Rees weak boson-gravitation relation Eq(2), and one comes from the Davies analysis , involving the Cosmological Microwave Background (CMB) wavelength. \textit{In the macro-physics side, with length unit $\lambdabar_e$, the Electron Compton reduced wavelength}, $6 \times$ the Hubble radius 13.812 billion light-years, Eq.(2), is tied to the bosonic critical dimension 26, while Bott reduction $\Delta$d = 8 leads firstly to  d = 18: it is the \textit{thermal photon} (CMB). This temperature $T \approx 2.725820805 Kelvin$, Eq. (31,) is identified to the common temperature of the couple Universe-Grandcosmos. It is tied to the the mammal wavelength through the Sternheimer scale factor $j$ (section 8.3); another Bott reduction leads to d = 10 (super-string dimension): it is the \textit{Hydrogen atom}, and finally to d = 2: the \textit{massive string}, about 2.1 GeV. For the number 24 of transverse dimensions, it is the \textit{Kotov length} (section 4.3), multiplied by a factor about 2$\pi a$, with $a \approx 137.036$. For d $\approx \Gamma$, the Atiyah constant (section 8.2), it is the \textit{galaxy group} radius, a characteristic cosmic length ($10^{6}$ light-years, section 2.1). For k $\approx e^{2}$, $y \approx 2e$, it is the \textit{Grandcosmos} radius (section 3). 
The Space-Time-Matter Holic dimension d = 30 (section 6) is tied to $c$ times the cosmic \textit{Supercycle period} (section 5). 
\textit{In the micro-physics side, with the same length unit $\lambdabar_e$}, Bott reductions from d = 30 lead to the \textit{gauge bosons}: d = 22 for the Grand Unification Theory (GUT) one, ($ 2.30\cdot10^{16}$ GeV), d = 14 for the weak one and d = 6 for the (\textit{massive}) gluons, about 8.6 MeV.
For the intermediary superstring value d = 10, there is the mean \textit{Pion}. For d $\approx \gamma \times \Gamma$, Y $\approx 495^2$ the square of the diminished Green-Schwarz string dimension (496 - 1), it is the \textit{Brout-Englert-Higgs boson} (125.175 GeV). For k $\approx 2e^e$, it is the \textit{topon}, the visible Universe wavelength, the space quantum, which identifies with the mono-radial unit length of the Bekenstein-Hawking Universe entropy (section 3).
   \textit{With unit $2\pi$ times the Nambu mass $m_N~=~am_e$}, d = 24 and 26 corresponds to the \textit{photon and graviton masses}, defined by the two-step holographic interaction\cite{Sanchez1}, section 7.4.}
This is the extrapolation towards smaller numbers of the Double Larger Number correlation.
The central dimension is d = 16, for a total of $2^7$ string dimensions in the Bott sequence.
This suggests a liaison with the Eddigton's matrix $16 \times 16$ \cite{Eddington}. 
    

\end{figure}

\end{document}
